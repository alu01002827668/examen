\documentclass[spanish,a4paper,10pt]{article}

\usepackage{latexsym,amsfonts,amssymb,amstext,amsthm,float,amsmath}
\usepackage[spanish]{babel}
\usepackage[utf8]{inputenc}
\usepackage[dvips]{epsfig}
\usepackage{doc}

\begin{document}
\title{SERIES DE TAYLOR}
\author{Jorge Antonio Herrera Alonso , Yessica Sabrina Gómez Buso , Elizabeth Hernández Martín}
\date{11 de mayo de 2014}

\maketitle

\section{Capítulo 1}
\subsection{\bf Motivación y objetivos}
En esta trabajo, vamos a tratar el tema de las Series Numéricas de Taylor.
Una serie de Taylor es una representación de una función como una infinita suma de términos.
Estos términos se calculan a partir de las derivadas de la función para un determinado valor de la variable (respecto de la cual se deriva), lo que involucra un punto específico sobre la función. Si esta serie está centrada sobre el punto cero, se le denomina serie de McLaurin.

\begin{enumerate}
  \item
    \bf Objetivo principal: 
        Implementación con Python del estudio de las series de Taylor.
  \item
    \bf Objetivo específico: 
        Como se desarrollan estas series numéricas para el caso $f(x)=sin(x)$
\end{enumerate}

\section{Capítulo 2}
\subsection{\bf Fundamentos teóricos}
Este tipo de representaciones son muy usadas en el estudio matemático, ya que tiene tres grandes ventajas ,como son:
\begin{enumerate}
 \item
  La derivación e integración de una de estas series se puede realizar término a término, que resultan operaciones triviales.
 \item
  Se puede utilizar para calcular valores aproximados de la función.
 \item
  Es posible demostrar que, si es viable la transformación de una función a una serie de Taylor, es la óptima aproximación posible.
\end{enumerate}

\subsection{\bf 2.1 Historia}
El filósofo eleata Zenón de Elea consideró el problema de sumar una serie infinita para lograr un resultado finito, pero lo descartó por considerarlo imposible: el resultado fueron las paradojas de Zenón. Posteriormente, Aristóteles propuso una resolución filosófica a la paradoja, pero el contenido matemático de esta no quedó resuelto hasta que lo retomaron Demócrito y después Arquímedes. Fue a través del método exhaustivo de Arquímedes que un número infinito de subdivisiones geométricas progresivas podían alcanzar un resultado trigonométrico finito.1 Independientemente, Liu Hui utilizó un método similar cientos de años después.2

En el siglo XIV, los primeros ejemplos del uso de series de Taylor y métodos similares fueron dados por Madhava de Sangamagrama.3 A pesar de que hoy en día ningún registro de su trabajo ha sobrevivido a los años, escritos de matemáticos hindúes posteriores sugieren que él encontró un número de casos especiales de la serie de Taylor, incluidos aquellos para las funciones trigonométricas del seno, coseno, tangente y arcotangente.

En el siglo XVII, James Gregory también trabajó en esta área y publicó varias series de Maclaurin. Pero en 1715 se presentó una forma general para construir estas series para todas las funciones para las que existe y fue presentado por Brook Taylor, de quién recibe su nombre.

Las series de Maclaurin fueron nombradas así por Colin Maclaurin, un profesor de Edinburgo, quién publicó el caso especial de las series de Taylor en el siglo XVIII.
\subsection{\bf 2.2 Serie de Taylor}
    $sin x = \sum_{n=0}^{\infin} \frac{f^{(n)}(a)}{n!} (x-a)^{n}, \forall x; n \in \mathbb{N}_0 $ 
\subsection{2.3 Serie de Maclaurin(Taylor alrededor de 0)}
    $sin x = \sum^{\infin}_{n=0} \frac{(-1)^n}{(2n+1)!} x^{2n+1}\quad, \forall x; n \in \mathbb{N}_0 $
\subsection{\bf 2.4 Aplicaciones}    
Además de la obvia aplicación de utilizar funciones polinómicas en lugar de funciones de mayor complejidad para analizar el comportamiento local de una función, las series de Taylor tienen muchas otras aplicaciones.

Algunas de ellas son: análisis de límites y estudios paramétricos de los mismos, estimación de números irracionales acotando su error, teorema de L'Hopital para la resolución de límites indeterminados, estudio de puntos estacionarios en funciones (máximos o mínimos relativos o puntos sillas de tendencia estrictamente creciente o decreciente), estimación de integrales, determinación de convergencia y suma de algunas series importantes, estudio de orden y parámetro principal de infinitésimos, etc.

\section{Capítulo 3}
\subsection{\bf Procedimiento experimental}
En esta sección vamos a ver la demostración de

Amplíe el programa \textsf{Python} que ha desarrollado para que el número de
subintervalos se pueda obtener también desde la línea de comandos.

\begin{thebibliography}{1}
\bibitem{python} Tutorial de Python. http://docs.python.org/2/tutorial/
\end{thebibliography}

\end{document}